\documentclass{article}

% pacakages
\usepackage{amsmath, amsfonts, amssymb}
\usepackage{stmaryrd}
\usepackage{fancyhdr}
\usepackage{lastpage}
\usepackage{lipsum}
\usepackage{graphicx}
\usepackage[ddmmyyyy]{datetime}
\usepackage{adjustbox}
\usepackage[a4paper, portrait, margin=20mm]{geometry} % définie le format de la page
\usepackage[explicit]{titlesec}
\usepackage{color, soul}
\setulcolor{red}

%personalized section style
\titleformat{\section}
{\Large\bfseries}
{\thesection}{1em}{\ul{#1}}

%code formatting
\usepackage{minted}
\usemintedstyle{manni}

%divers commands
\newcommand{\bb}[1]{\mathbb{#1}}
\newcommand{\encadrer}[1]{\fbox{color=red
    \begin{minipage}{0.90\textwidth}
        #1
    \end{minipage}
}}
\renewcommand{\thesection}{\Roman{section}} % Roman numerals for sections
\setlength{\headheight}{12.5pt}
\newcommand{\image}[3]{ %command to insert image
    \begin{minipage}[t]{\linewidth}
        #1
              \adjustbox{valign=t}{%
                \includegraphics[width=#2\linewidth]{#3}%
              }
    \end{minipage}}

\renewcommand*{\thesection}{\arabic{section}}
\renewcommand*{\thesubsection}{\arabic{section}.\arabic{subsection}}

%page numerotation
\pagestyle{fancy}
\fancyhf{}
\renewcommand{\headrulewidth}{0pt}
\fancyfoot[R]{\thepage/\pageref{LastPage}}

%document info
\makeatletter
\title{DM2 - Mines MP 2019}
\date{\today}
\newcommand{\matiere}{Informatique Option}
\newcommand{\classe}{MP\textsuperscript{*} }
\author{Arsène MALLET}

%header
\fancypagestyle{firstpage}{
    \fancyhead[L]{\@author}
    \fancyhead[C]{\classe - \matiere}
    \fancyhead[R]{\@date}
}


\begin{document}

\thispagestyle{firstpage}

\begin{center}
    \huge\bfseries{\@title}
\end{center}

\section{Premiers exemples}

\begin{enumerate}
    \item Le langage reconnu par l'automate $A_1$ est l'ensemble des mots de taille impaire.
    \item Le langage reconnu par l'automate $A_2$ est l'ensembles des mots contenant un nombre impair de $b$.
    \item $L(A_1) = (a \cdot a|b \cdot b|a \cdot b|b \cdot a)^* \cdot (a|b)$
    \item $L(A_1) = (a|b \cdot a^* \cdot b)^*\cdot b\cdot a^*$
    \item \begin{minted}{ocaml}
    let a2 = 2, [|0, 1 ; 1, 0|], [|false; true|]  ;;
    \end{minted}
\end{enumerate}

\section{\'Etats accessibles d'un automate}

\begin{enumerate}
    \setcounter{enumi}{5}
    \item \begin{minted}{ocaml}
    let numero n 
    \end{minted}
\end{enumerate}


\end{document}