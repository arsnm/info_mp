\documentclass{article}

% pacakages
\usepackage{amsmath, amsfonts, amssymb}
\usepackage{stmaryrd}
\usepackage{fancyhdr}
\usepackage{lastpage}
\usepackage{lipsum}
\usepackage{graphicx}
\usepackage[ddmmyyyy]{datetime}
\usepackage{adjustbox}
\usepackage[a4paper, portrait, margin=20mm]{geometry} % définie le format de la page
\usepackage[explicit]{titlesec}
\usepackage{color, soul}
\setulcolor{red}

%personalized section style
\titleformat{\section}
{\Large\bfseries}
{\thesection}{1em}{\ul{#1}}

%code formatting
\usepackage[outputdir=.aux]{minted}
\usemintedstyle{manni}

%divers commands
\newcommand{\bb}[1]{\mathbb{#1}}
\newcommand{\encadrer}[1]{\fbox{color=red
    \begin{minipage}{0.90\textwidth}
        #1
    \end{minipage}
}}
\renewcommand{\thesection}{\Roman{section}} % Roman numerals for sections
\setlength{\headheight}{12.5pt}
\newcommand{\image}[3]{ %command to insert image
    \begin{minipage}[t]{\linewidth}
        #1
              \adjustbox{valign=t}{%
                \includegraphics[width=#2\linewidth]{#3}%
              }
    \end{minipage}}

%page numerotation
\pagestyle{fancy}
\fancyhf{}
\renewcommand{\headrulewidth}{0pt}
\fancyfoot[R]{\thepage/\pageref{LastPage}}

%document info
\makeatletter
\title{TD - Langages Rationnels}
\date{\today}
\newcommand{\matiere}{Informatique Option}
\newcommand{\classe}{MP\textsuperscript{*} }
\author{Arsène MALLET}

%header
\fancypagestyle{firstpage}{
    \fancyhead[L]{\@author}
    \fancyhead[C]{\classe - \matiere}
    \fancyhead[R]{\@date}
}


\begin{document}

\thispagestyle{firstpage}

\begin{center}
    \huge\bfseries{\@title}
\end{center}

\section{R\`egles Op\'eratives}

\section{Petites Questions}

\begin{enumerate}
    \item $c^*ac^*bc^* | c^*bc^*ac^*$
    \item $\{0, 1 \}^* \infty$
    \item $((a + \varepsilon)(b + \varepsilon))^* (a + \varepsilon)$
    \item Soit $e = (bc + b)^*$, $eaeae$ marche.
    \item $L(\frac{1}{6}) = \varepsilon |16^*$ ; $L(\frac{1}{7}) = (142857)*(e|\varepsilon)$
\end{enumerate}

\section{Distance de Hamming}

\begin{enumerate}
    \item Positif, sym\'etrique, nulle : $\forall i, u_i = v_i \implies u - v = 0$, IT ...
    \item \begin{minted}{ocaml}
        let dist u v =
          let d = ref 0 in 
          for i = 0 to String.length u - 1 do 
            if u.(i) <> v.(i) then incr d
          done ;
          !d ;;
    \end{minted}
    \item $\mathcal{H}(L) = 0^*10^*1^* | 0^*1^*01^* | \varepsilon$ 
    \item \begin{itemize}
        \item[--] $f(\varnothing) = \varnothing$ ;
            $f(\varepsilon) = \varepsilon$ ; 
            si $a \in \sum$, $f(a) = \sum $
        \item[--] $f(e_1|e_2) = f(e_1)f(e_2)$
        \item[--] $f(e_1e_2) = f(e1)e2 | e_1f(e_2)$
        \item[--] $f(e^*) = e^*f(e)e^*$
    \end{itemize}
    $P_n$ : Si $e$ est une expression rationnel de taille $n$ alors $\mathcal{H}$ est rationel.
    \begin{itemize}
        \item $P_1$ est vraie : \begin{itemize}
            \item $\mathcal{H}(\varnothing) = \varnothing$
            \item $\mathcal{H}(\epsilon) = \epsilon$ ...
        \end{itemize}
        \item $\forall k \leq n, P_k \implies P_{n + 1}$, soit $e$ une expression de talle $n + 1$ \newline
            Si $e = e_1|e_2$, on applique $P_k$ sur $e_1$ et $e_2$, ce qui nous donne $e'_1$ et $e'_2$ d'où $\mathcal{H}(e) = e'_1|e'_2$
    \end{itemize}
    \item \begin{minted}{ocaml}
        let rec f e = match e with
            |Vide -> Vide
            |Epsilon -> Epsilon
            |L a -> Union(L 0, L 1)
            |Union(e_1, e_2) -> Union(f e_1, f e_2)
            |Concat(e_1, e_2) -> Union(Concat(f e_1, e_2), Concat(e_1, f e_2))
            |Etoile e -> Concat(Concat(Etoile e, f e), Etoile e)
    \end{minted}
\end{enumerate}

\section{Hauteur d'\'etoile}

\begin{enumerate}
    \item $h((ba^*b)^*) = 1 + h(ba^*b) = 1 + max(h(ba^*), h(b)) = 1 + max(h(a^*), h(b)) = 2$
    \item \begin{minted}{ocaml}
        let rec h expr = match expr with
            |Vide |Epsilon |L(_) -> 0
            | Union(a, b) |Concat(a, b) -> max(h(a), h(b))
            | Etoile(a) -> 1 + h(a)
    \end{minted}
    \item Les languages d'hauteur d'\'etoile 0 contiennent uniquement un nombre fini de mots.
\end{enumerate}

\section{Cl\^oture par sous-mot}

\section{Utilisisation de la programmation dynamique sur les mots}

\section{Lemme d'Arden}


\end{document}