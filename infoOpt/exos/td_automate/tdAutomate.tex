\documentclass{article}

% pacakages
\usepackage{amsmath, amsfonts, amssymb}
\usepackage{stmaryrd}
\usepackage{fancyhdr}
\usepackage{lastpage}
\usepackage{lipsum}
\usepackage{graphicx}
\usepackage[ddmmyyyy]{datetime}
\usepackage{adjustbox}
\usepackage[a4paper, portrait, margin=20mm]{geometry} % définie le format de la page
\usepackage[explicit]{titlesec}
\usepackage{color, soul}
\setulcolor{red}

%personalized section style
\titleformat{\section}
{\Large\bfseries}
{\thesection}{1em}{\ul{#1}}

%code formatting
\usepackage{minted}
\usemintedstyle{manni}

%divers commands
\newcommand{\bb}[1]{\mathbb{#1}}
\newcommand{\encadrer}[1]{\fbox{color=red
    \begin{minipage}{0.90\textwidth}
        #1
    \end{minipage}
}}
\renewcommand{\thesection}{\Roman{section}} % Roman numerals for sections
\setlength{\headheight}{12.5pt}
\graphicspath{{./img}}
\newcommand{\image}[3]{ %command to insert image
    \begin{minipage}[t]{\linewidth}
        #1
              \adjustbox{valign=t}{%
                \includegraphics[width=#2\linewidth]{#3}%
              }
    \end{minipage}}

%page numerotation
\pagestyle{fancy}
\fancyhf{}
\renewcommand{\headrulewidth}{0pt}
\fancyfoot[R]{\thepage/\pageref{LastPage}}

%document info
\makeatletter
\title{TD - Automates}
\date{\today}
\newcommand{\matiere}{Informatique Option}
\newcommand{\classe}{MP\textsuperscript{*} }
\author{Arsène MALLET}

%header
\fancypagestyle{firstpage}{
    \fancyhead[L]{\@author}
    \fancyhead[C]{\classe - \matiere}
    \fancyhead[R]{\@date}
}


\begin{document}

\thispagestyle{firstpage}

\begin{center}
    \huge\bfseries{\@title}
\end{center}

\section{Algorithmes de d\'eterminisation}

\begin{enumerate}
    \item \image{}{0.5}{I-1/I-1.pdf}
    \item $L = a((b|a)bb^*a)^*$
\end{enumerate}

\section{Cl\^oture des langages reconnaissables}

\begin{enumerate}
    \item \begin{itemize}
        \item[] Soit $A = (\Sigma, Q, I, F, \delta)$ tq $L = L(A)$
        \item[] Soit $A' = (\Sigma, Q, I, F, \delta')$ où $\delta'(q, a) = \{ q' |  \delta(q', a) = q \}$
        \item[] \image{}{0.2}{II-1/II-1.pdf}
        \item Mq : $m \in L(\tilde A) \Leftrightarrow ... \Leftrightarrow m \in \widetilde{L(A)}$
        \item[] $m = m_1...m_n \in L(\tilde A)$
        \item[] $\Leftrightarrow \exists \text{ chemin } q_0 \leftarrow q_1 ... \leftarrow q_n \text{ dans } \tilde A$
        \item[] $\Leftrightarrow \exists \text{ chemin } q_0 \rightarrow q_1 ... \rightarrow q_n \text{ dans } A$
        \item[] $\Leftrightarrow m_1...m_n \in L(A)$
        \item[] $\Leftrightarrow \tilde m \in L(A)$
        \item[] $\Leftrightarrow \in \widetilde{L(A)}$
    \end{itemize}
    \item \begin{itemize}
        \item[] Soit $A = (\Sigma, Q, I, F, \delta)$ tq $L = L(A)$
        \item[] Soit $A' = (\Sigma, Q, I, F', \delta)$ où $F'$ est l'ensembles des états co-accessibles dans $A$,
        \item[] $m \in L(A') \Leftrightarrow m \in Pref(L)$
        \item[] Soit $A'' = (\Sigma, Q, I', F, \delta)$ où $I'$ est l'ensembles des états accessibles dans $A$
        \item[] Soit $A''' = (\Sigma, Q, I', F', \delta)$
    \end{itemize}
    \item \begin{itemize}
        \item $H_n$ : "Si $e$ est une expression rationnelle de taille $n$ alors il existe une expression rationnelle pour $Pref(e)$"
        \item[-] $H_1$, $e = \varnothing, \varepsilon, a \in \Sigma$
        \item[] $Pref(\varnothing) = \varnothing$, $Pref(\varepsilon) = \varepsilon$, $Pref(a) = \varepsilon|a$
        \item[-] Soit $n \in \bb{N}^*$, supposons $H_k$, $\forall k \leq n$
        \item[] Soit $e$ expression rationnelle de taille $n + 1$:
        \item[] \begin{enumerate}
            \item Si $e = e_1 | e_2$ : $Pref(e) = Pref(e_1)|Pref(e_2)$, une expression rationnelle
            \item Si $e = e_1e_2$: $Pref(e) = Pref(e_1)|e_1Pref(e_2)$, une expression rationnelle
            \item Si $e = e_1^*$: $Pref(e) = e1^*Pref(e_1)$, une expression rationnelle
        \end{enumerate}
        \item $Suff(L) = \widetilde{Pref(\tilde L)}$, or $\tilde L$ est rationnelle d'apres cours, donc $\widetilde{Pref(\tilde L)}$ est également rationnelle d'après ce que l'on vient de démontrer.
        \item $Fact(L) = Suff(Pref(L))$ 
    \end{itemize}
\end{enumerate}

\section{Reconnaissable ou non ?}

\begin{enumerate}
    \item \image{}{0.2}{III-1/III-1.pdf} Ainsi, il existe un automate $A_1$ tel que $L(A_1) =L_1$ donc $L_1$ est reconnaissable.
    \item \image{}{0.2}{III-2/III-2.pdf} De même, il existe un automate $A_2$ tel que $L(A_2) =L_2$ donc $L_2$ est reconnaissable.
    \item \begin{itemize}
        \item[] Même démo que pour $\{a^nb^n | n \in \bb{N}\}$ :
        \item[] $\{a^nb^n | n \in \bb{N}\} = L_3 \cap a^*b^* \implies L_3$ non reconnaissable.
    \end{itemize}
    \item \image{\centering}{0.8}{III-4/III-4.pdf}
    \item \begin{itemize}
        \item[] Supposons $L_5$ reconnaissable,
        \item[] Soit $n$ l'entier donné par le Lemme de l'\'Etoile,
        \item[] Il existe p, nombre premier supérieur à $n$
        \item[] Soit $u = a^p$. $u \in L_5$ et $|u| \geq n$ donc :
        \item[] $\exists x, y, z \in \Sigma^*$ tq $u = xyz$,  $|xy| \leq n$, $y \neq \varepsilon$
        \item[] $\exists i, j k$ tq $x = a^i$, $y = a^j$, $z = a^k$
        \item[] $xy^{i+k}z = a^ia^{j(i + k)}a^k = a^{(i + k)(1 + j)} \notin L_5$ car $k \geq 2$
        \item[] \textbf{Absurde}, donc $L_5$ non reconnaissable.
    \end{itemize}
\end{enumerate}

\section{Algorithmes sur les automates}

\section{Oral ENS info}

\section{Algorithme KMP}

\section{R\'esiduel}
\end{document}