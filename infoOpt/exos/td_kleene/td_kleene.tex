\documentclass{article}

% pacakages
\usepackage[french]{babel}
\usepackage{caption}
\usepackage[T1]{fontenc}
\usepackage{amsmath, amsfonts, amssymb}
\usepackage{stmaryrd}
\usepackage{fancyhdr}
\usepackage{lastpage}
\usepackage{lipsum}
\usepackage{cprotect}
\usepackage{graphicx}
\usepackage[ddmmyyyy]{datetime}
\usepackage{adjustbox}
\usepackage[a4paper, portrait, margin=20mm]{geometry} % define the page format
\usepackage[explicit]{titlesec}
\usepackage{color, soul}
\setulcolor{red}

%reference for the items of 'enumerate'
\usepackage{enumitem, hyperref}
\makeatletter
\def\namedlabel#1#2{\begingroup
    #2%
    \def\@currentlabel{#2}%
    \phantomsection\label{#1}\endgroup
}
\makeatother

%enumerate style
\renewcommand{\labelenumi}{\arabic{enumi}.}

%personalized section style
\titleformat{\section}
{\Large\bfseries}
{\thesection}{1em}{\ul{#1}}

%code formatting
\usepackage[outputdir=.aux]{minted}
\usemintedstyle{manni}

%divers commands
\newcommand{\bb}[1]{\mathbb{#1}}
\newcommand{\encadrer}[1]{\fcolorbox{red}{white}{
    \begin{minipage}{\textwidth}
        #1
    \end{minipage}
}}
\renewcommand{\thesection}{\Roman{section}} % Roman numerals for sections
\setlength{\headheight}{12.5pt}

\graphicspath{ {./img/} } % define path to img 
\newcommand{\image}[3]{ %command to insert image
    \begin{minipage}[t]{\linewidth}
        #1
        \adjustbox{valign=t}{
            \includegraphics[width=#2\linewidth]{#3}
        }
    \end{minipage}}

%page numerotation
\pagestyle{fancy}
\fancyhf{}
\renewcommand{\headrulewidth}{0pt}
\fancyfoot[R]{\thepage/\pageref{LastPage}}

%document info
\makeatletter
\title{TD Rationnel $\Longleftrightarrow$ Reconnaissable}
\date{\today}
\newcommand{\matiere}{Informatique Option}
\newcommand{\classe}{MP\textsuperscript{*} }
\author{Arsène MALLET}

%header
\fancypagestyle{firstpage}{
    \fancyhead[L]{\@author}
    \fancyhead[C]{\classe - \matiere}
    \fancyhead[R]{\@date}
}


\begin{document}

\thispagestyle{firstpage}

\begin{center}
    \huge\bfseries{\@title}
\end{center}

\section{Exercice CCP}
\begin{enumerate}
    \item \image{\centering}{0.25}{I-1/I-1.pdf}
\end{enumerate}
\section{Langage local, lin\'eaire, automate de Glushkov}
\begin{enumerate}
    \item \begin{itemize}
        \item \underline{Linéarisation} : $L = (ab + c)^*d$
        \item $P(L) = \{a, c, d \}$
        \item $S(L) = \{d\}$
        \item $F(L) = \{ab, ba, ca, bd, cd, bc, cc\}$
    \end{itemize}
    \image{\centering}{0.8}{II-1/II-1.pdf}
\end{enumerate}

\section{Stabilit\'e des langages rationnels}

\begin{enumerate}
    \item Non, $\underbrace{\varnothing}_{\text{rat.}} \subset \underbrace{\{a^nb^n \to n \in \bb{N} \}}_{\text{non-rat.}}$.
    \item Non plus, $\underbrace{\{a^nb^n \to n \in \bb{N} \}}_{\text{non-rat.}} \subset \underbrace{\Sigma^*}_{\text{rat.}}$.
    \item Oui, par définition d'un langage rationnel.
    \item Non, $L = \{\text{palyndrome}\}$ non rat. mais $L^* = \Sigma^*$ est rationnel.
    \item Oui, voir démonstration cours (inversé états finals et initiaux).
    \item Oui, par récurrence et stabilité d'un langage par union.
    \item Non, $\cup_{k \in \bb{N}} \underbrace{\{a^kb^k\}}_{\text{rat.}} = \underbrace{\{a^nb^n \to n \in \bb{N}\}}_{\text{non-rat.}}$.
    \item Oui, car $\cap L_k =$ Complementaire $\cup$ complementaire $L_k$ (par 5 et 6).
    \item Non, par l'absurde.
\end{enumerate}

\section{Calcul des ensembles $P(L)$, $S(L)$ et $F(L)$}

\section{Reconnaissable $\implies$ rationnel avec le lemme d'Arden}

\begin{enumerate}
    \item 
    \begin{itemize}
        \item \underline{Analyse :}
        \begin{align*}
            L &= AL \cup B \\
            &= A(AL \cup B) \cup B \\
            &= A^2L \cup AB \cup B \\
            &= A^kL \cup A^{k-1}B \cup ... AB \cup B \\
            \forall k, \; L &= \cup_{i = 0}^{k - 1} (A^iB) \cup A^kL \\
            &\forall i, \; A^iB \subset L
         \end{align*}
         Donc $A^*B \subset L$
        \item Finir de recopier récurrence...
    \end{itemize}
    \item \begin{align}
        L_0 = aL_1 \cup bL_0 \cup \varepsilon \\
        L_1 = aL_2 \\
        L_2 = \varepsilon \cup bL_0
    \end{align}
    Remplacer (2) dans (1): $L$ \\
    Remplacer dans (3) : $L$ \\
    Lemme d'Arden $\implies L_0 = (a^2b \cup b)^*(a^2 \cup \varepsilon)$
\end{enumerate}


\end{document}