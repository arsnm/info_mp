\documentclass{article}

% pacakages
\usepackage{amsmath, amsfonts, amssymb}
\usepackage{stmaryrd}
\usepackage{fancyhdr}
\usepackage{lastpage}
\usepackage{lipsum}
\usepackage{graphicx}
\usepackage[ddmmyyyy]{datetime}
\usepackage{adjustbox}
\usepackage[a4paper, portrait, margin=20mm]{geometry} % définie le format de la page
\usepackage[explicit]{titlesec}
\usepackage{color, soul}
\setulcolor{red}

%personalized section style
\titleformat{\section}
{\Large\bfseries}
{\thesection}{1em}{\ul{#1}}

%code formatting
\usepackage[outputdir=.aux]{minted}
\usemintedstyle{manni}

%divers commands
\newcommand{\bb}[1]{\mathbb{#1}}
\newcommand{\encadrer}[1]{\fbox{
    \begin{minipage}{0.90\textwidth}
        #1
    \end{minipage}
}}
\renewcommand{\thesection}{\Roman{section}} % Roman numerals for sections
\setlength{\headheight}{12.5pt}
\newcommand{\image}[3]{ %command to insert image
    \begin{minipage}[t]{\linewidth}
        #1
              \adjustbox{valign=t}{%
                \includegraphics[width=#2\linewidth]{#3}%
              }
    \end{minipage}}

%page numerotation
\pagestyle{fancy}
\fancyhf{}
\renewcommand{\headrulewidth}{0pt}
\fancyfoot[R]{\thepage/\pageref{LastPage}}

%document info
\makeatletter
\title{TD - Fil de priorité et Tas}
\date{\today}
\newcommand{\matiere}{Informatique}
\newcommand{\classe}{MP\textsuperscript{*} }
\author{Arsène MALLET}


%header
\fancypagestyle{firstpage}{
    \fancyhead[L]{\@author}
    \fancyhead[C]{\classe - \matiere}
    \fancyhead[R]{\@date}
}


\begin{document}

\thispagestyle{firstpage}

\begin{center}
    \huge\bfseries{\@title}
\end{center}

\section{Un tas de question}

\begin{enumerate}

\item \image{\raggedright}{0.6}{img/I_1.pdf}

\item

\item \begin{minted}{ocaml}
    let rec is_tas_max heap = 
        test = ref true in
        for i = 0 to heap.n - 1 do
            if a.(i) > a.((i-1)/2) then
            test := false
        done;
        !test ;;
\end{minted}

\item \begin{minted}{ocaml}
    type 'a heap = {a : 'a array ; mutable n : int}
    type 'a tree = E | N of 'a * 'a tree * 'a tree

    let heap_to_tree heap=
        let rec aux i = (*renvoie le sous-arbre enraciné en heap.a *)
        if i >= heap.n
            then E
            else N(heap.a.(i), aux(2*i + 1), aux(2 * (i + 1))) in
        aux 0 ;;
\end{minted}


\item \begin{minted}{ocaml}
    let rec fusion l1, l2 = match l1, l2 with
        |[], _ -> l2
        |_, [] -> l1
        |e1::q1, e2::q2 -> if e1 < e2 then
                                e1::fusion q1 l2
                            else
                                e2::fusion l1 q2 ;;
\end{minted}

La complexité est $O(n+m)$

\image{\centering}{0.8}{img/I_5.pdf}

\begin{minted}{ocaml}
    let rec etape ll = match ll with
        |l1::l2::q -> (fusion l1 l2)::etape q
        |_ -> ll ;;

    let rec kfusion ll = match ll with
        |[] -> []
        |[l] -> l
        |_ -> kfusion (etape ll) ;;
\end{minted}

\end{enumerate}

\section{Compression de Huffman}

\begin{enumerate}
    
\item \begin{minted}{ocaml}
    let rec read arb lst =
    match arb with
    | F(chr) -> chr, lst
    | N(g, d) -> match lst with
                    |0::t -> read g t
                    |1::t -> read d t
                    |_ -> failwith "liste non valide"
\end{minted}

\item \begin{minted}{ocaml}
    let rec decode arb lst =
        match (read arb lst) with
            |chr, [] -> chr::[]
            |chr, lst_end -> chr::(decode arb lst_end) ;;
\end{minted}

\item \image{\raggedright}{0.45}{img/II_3.pdf}

\item \begin{minted}{ocaml}
    let to_huffman freq =
        let fp = create () in
        let rec fill freq=
        match freq with
        |[] -> ()
        |(f, chr)::t -> add fp (f, F(chr)) ; fill t in
        fill freq;
        while len_over_2 fp do
        let (f1, a1), (f2, a2) = extract_min fp, extract_min fp in
            add fp (f1 + f2, N(a1, a2))
        done;
        snd (extract_min fp) ;;
\end{minted}

\item \begin{minted}{ocaml}
    let to_dict arb = 
        let dict = Hashtbl.create 100 in
        let rec aux arb dict code =
            match arb with
            |F(chr) -> Hashtbl.add dict chr (List.rev code)
            |N(g, d) -> aux g dict (0::code) ; aux d dict (1::code) 
        in
        aux arb dict [] ; dict ;;
\end{minted}

\item \begin{minted}{ocaml}
    let rec code lst dict =
        match lst with
        |[] -> []
        |h::t -> (Hashtbl.find dict h) @ code t dict
\end{minted}
\end{enumerate}

\section{Arbretas}

\begin{enumerate}

\item \begin{minted}{ocaml}
    let swap t i j =
        let tmp = t.(j) in
        t.(j) <- t.(i); t.(i) <- tmp
    ;;
\end{minted}

\item La fonction \texttt{shuffle t} permutte tous les éléments du tableau \texttt{t} avec un autre éléments du tableau
choisi au hasard dans les indices inférieurs. Après avoir terminé la boucle \texttt{for}, le tableau a donc subit une permutation
aléatoires.

\item \image{\raggedright}{0.25}{img/III_3.pdf}

\item Soit $n \in \bb{N}$, le nombre d'éléments de l'abretas, on note $e_1, ..., e_n \in \bb{N}^2$ ces éléments.
Montrons par récurrence forte sur $n$ l'unicité de l'arbretas. \\
On note $P(n)$ l'assertion "Un arbretas à $n$ éléments distincts est unique" \\
\fbox{Initialisation} : Si $n = 1$, alors l'abretas est évidemment unique. \\
\fbox{Hérédité} : Soit $n \in \bb{N}$, supposons $P(n), P(n-1), ..., P(1)$, montrons $P(n+1)$ :
On choisi $i$ tel que $si e_i = (x, p)$, alors $p$ est la priorité minimum, alors par propriété de l'abretas, $e_i$ est la racine,
on peut alors creer les sous-arbres gauche et droite, de taille au plus $n$ qui sont donc uniques par {\it hypothèse de récurrence}. \\
D'où l'unicité. 

\item \begin{minted}{ocaml}
    let rotd treap = match treap with
    |N(r, N(gr, gg, gd), d) -> N(gr, N(r, gd, d), gg)
    |_ -> treap
\end{minted}


\item \image{\raggedright}{}{img/III_6.pdf}

\item \begin{minted}{ocaml}
    let prio tree = match tree with
    |V -> max_int
    |N((_, p), _, _) -> p
    ;;
\end{minted}

\item \begin{minted}{ocaml}
    let rec add treap e = 
      let elem, _ = e in
      match treap with
      |V -> N(e, V, V)
      |N((x, p), g, d) -> if elem >= x then 
                              let d_upt = add d e in
                                  if (prio d_upt) < p then
                                    rotg (N((x,p), g, d_upt))
                                  else N((x,p), g, d_upt)
                          else 
                              let g_upt = add g e in
                              if (prio g_upt) < p then
                                  rotd (N((x,p), g_upt, d))
                                else N((x,p), g_upt, d)
                                ;;
\end{minted}

\item \begin{minted}{ocaml}
    let rec del treap e = match treap with
    |V -> V
    |N((x,p), g, d) -> if e > x then
                        N((x,p), g, (del d e))
                      else if e < x then
                        N((x,p), (del g e), d)
                      else match g,d with
                        |V, V -> V
                        |V, f |f, V -> f
                        |_ -> if prio g < prio d then
                              let treap_rot = rotd(treap) in
                                del treap_rot e
                              else let treap_rot = rotg(treap) in
                                del treap_rot e
                            ;;
\end{minted}
\end{enumerate}
\end{document}