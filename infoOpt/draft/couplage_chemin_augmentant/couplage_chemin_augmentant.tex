\documentclass[varwidth = 10cm]{standalone}
\fontfamily{'Hack Nerd Font Mono'}


\begin{document}

{Démonstration} \\
Si $G = (V, E)$, soit $ v \in V $

Supposons que :
\begin{itemize}
    \item $ v \in e_1$ où $e_1 \in M \Delta P $
    \item $ v \in e_2$ où $e_2 \in M \Delta P $
\end{itemize}

On ne peut pas avoir $ e_1 \in M $ et $ e_2 \in M $ (car $M$ couplage) ni $e_1 \in P \backslash M$ et $e_2 \in P \backslash M$ (car ce ne serait pas alternant).
Par symétrie, supposons $ e_1 \in M \backslash P $ et $ e_2 \in P \backslash M $

\begin{enumerate}
    \item Si $v$ est extrémité de P: \textbf{absurde} car $v$ est libre ($P$ augmentant)
    mais $v$ adjacent à $e_1 \in M$    
    \item $ \exists e_3 \neq e_2 \in P$ adjacent à $v$, $P$ est augmentant et $e_2 \notin M$ donc $e_3 \in M$
    \textbf{absurde} car $v$ adjacent à 2 arêtes de $M$
\end{enumerate}

\end{document}