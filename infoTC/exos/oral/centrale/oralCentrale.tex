\documentclass{article}

% packages
\usepackage[french]{babel}
\usepackage{caption}
\usepackage[T1]{fontenc}
\usepackage{amsmath, amsfonts, amssymb}
\usepackage{stmaryrd}
\usepackage{fancyhdr}
\usepackage{lastpage}
\usepackage{lipsum}
\usepackage{graphicx}
\usepackage[ddmmyyyy]{datetime}
\usepackage{adjustbox}
\usepackage[a4paper, portrait, margin=20mm]{geometry} % define the page format
\usepackage[explicit]{titlesec}
\usepackage{color, soul}
\setulcolor{red}

%reference for the items of 'enumerate'
\usepackage{enumitem, hyperref}
\makeatletter
\def\namedlabel#1#2{\begingroup
    #2%
    \def\@currentlabel{#2}%
    \phantomsection\label{#1}\endgroup
}
\makeatother

%personalized section style
\titleformat{\section}
{\Large\bfseries}
{\thesection}{1em}{\ul{#1}}

%counters
\newcounter{exo}
\setcounter{exo}{0}
\newcommand{\exo}{%
{\stepcounter{exo}{\Large\textbf{Exercice \arabic{exo}}%
\newline}}}

\newcounter{question}
\setcounter{question}{0}
\newcommand{\question}{%
{\stepcounter{question}{\arabic{question}. \vspace{2mm}}}}

\newcounter{subquestion}
\setcounter{subquestion}{0}
\newcommand{\subquestion}{%
{\stepcounter{subquestion}{(\alph{subquestion})\vspace{2mm}}}}

%code formatting
\usepackage[cache=false, outputdir=.aux]{minted}
\usemintedstyle{manni}

%divers commands
\newcommand{\bb}[1]{\mathbb{#1}}
\newcommand{\encadrer}[1]{\fbox{color=red
    \begin{minipage}{0.90\textwidth}
        #1
    \end{minipage}
}}
% \renewcommand{\thesection}{\Roman{section}} % Roman numerals for sections
\setlength{\headheight}{12.5pt}

\newcommand{\reset}[1]{\setcounter{#1}{0}}

\graphicspath{ {./img/} } % define path to img 
\newcommand{\image}[3]{ %command to insert image
    \begin{minipage}[t]{\linewidth}
        #1
        \adjustbox{valign=t}{
            \includegraphics[width=#2\linewidth]{#3}
        }
    \end{minipage}}

%page numerotation
\pagestyle{fancy}
\fancyhf{}
\renewcommand{\headrulewidth}{0pt}
\fancyfoot[R]{\thepage/\pageref{LastPage}}

%document info
\makeatletter
\title{Exercices Oral Centrale}
\date{\today}
\newcommand{\matiere}{Informatique}
\newcommand{\classe}{MP\textsuperscript{*} }
\author{Arsène MALLET}

%header
\fancypagestyle{firstpage}{
    \fancyhead[L]{\@author}
    \fancyhead[C]{\classe - \matiere}
    \fancyhead[R]{\@date}
}


\begin{document}

\thispagestyle{firstpage}

\begin{center}
    \huge\bfseries{\@title}
\end{center}

\exo

\question Soit $A,\;B \in D_n$, $\lambda \in \bb{R}$, on a clairement que $A + \lambda B$ est de diagonale nulle, donc $A + \lambda B \in D_n$.
On a clairement $D_n \subset E_n$, donc $D_n$ est un sous-espace vectoriel de $E$. La dimension de $D_n$ est $n(n - 1)$ et une base de $D_n$ est
$(E_{i,j})_{i \neq j \in \llbracket 1, n \rrbracket^2}$.

\question Soit $A,\;B \in T_n$, $\lambda \in \bb{R}$, on a clairement que $A + \lambda B$ est de trace nulle, donc $A + \lambda B \in T_n$.
On a clairement $T_n \subset E_n$, donc $T_n$ est un sous-espace vectoriel de $E$. La dimension de $T_n$ est $n^2 - 1$ et une base est $(E_{i,j})_{i \neq j \in \llbracket 1, n \rrbracket} \cup (E_{i,i} - E_{n, n})_{i \in \llbracket 1, n - 1 \rrbracket}$

\question \subquestion
\begin{minted}[breaklines]{python}
    def diag(n):
        return np.diag(np.random.rand(n))
\end{minted}

\subquestion
\begin{minted}[breaklines]{python}
    def matrice(A, B):
        return np.dot(A, B) - np.dot(B, A)
\end{minted}

\subquestion
\begin{minted}[breaklines]{python}
    n = 5
    A = diag(n)
    B = np.random.rand(n, n)
    print(matrice(A, B))
\end{minted}

La matrice obtenue est une matrice de diagonale nulle.

\reset{subquestion}

\question Soit $A$ une matrice diagonale et $B \in \mathcal{M}_n(\bb{R})$, montrons que $AB -  BA \in D_n$
Notons $A = \begin{pmatrix}
    \lambda_1 &  & (0) \\
     & \ddots &  \\
    (0) &  & \lambda_n
    \end{pmatrix}  $ et $B = (b_{i,j})$.
    On a alors si $i \in \llbracket 1, b \rrbracket$, $(ab)_{i,j} = \lambda_i b_{i,i}$
    De même, $(ba)_{i,i} = \lambda_i b_{i, i}$, d'où $(ab - ba)_{i, i} = 0$. Ainsi $AB - BA \in D_n$
\question Montrons qu'il n'est pas possible d'avoir $Card(S_M) = 1$. En effet si $(A, B) \in S_M$, alors $(-B, A)$ et $(B, -A)$ appartiennent aussi à $S_M$
Donc $|S_m = 1| \iff A = B = -A \iff A = B = (0)$ \textit{i.e.} or si $A = B = (0)$, $AB - BA = (0)$, or si $M = (0)$
alors $S_M = \{(A, B) | AB = BA\} \neq \{0\}$. Ainsi il n'existe pas de $M$ tel que $|S_M| = 1$ \\

\reset{question}

\exo

\question \begin{minted}[breaklines]{python}
    univers = [-1, 1]

    def temps(a, b):
    n = -1
    S_n = 0
    while -a <= S_n <= b:
        S_n += univers[np.random.randint(0, 2)]
        n += 1
    return n
\end{minted}

\question \begin{minted}[breaklines]{python3}
    def moyenne(a, b):
        return np.average(np.array([temps(5, 7) for _ in range(10000)]))
    \end{minted}
Après plusieurs exécutions, on obtient une moyenne fini, donc $T$ prend des valeurs finies.
C'est en tous cas la conjecture que l'on peut faire.


\end{document}